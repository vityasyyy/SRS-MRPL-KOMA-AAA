%Copyright 2014 Jean-Philippe Eisenbarth
%This program is free software: you can 
%redistribute it and/or modify it under the terms of the GNU General Public 
%License as published by the Free Software Foundation, either version 3 of the 
%License, or (at your option) any later version.
%This program is distributed in the hope that it will be useful,but WITHOUT ANY 
%WARRANTY; without even the implied warranty of MERCHANTABILITY or FITNESS FOR A 
%PARTICULAR PURPOSE. See the GNU General Public License for more details.
%You should have received a copy of the GNU General Public License along with 
%this program.  If not, see <http://www.gnu.org/licenses/>.

%Based on the code of Yiannis Lazarides
%http://tex.stackexchange.com/questions/42602/software-requirements-specification-with-latex
%http://tex.stackexchange.com/users/963/yiannis-lazarides
%Also based on the template of Karl E. Wiegers
%http://www.se.rit.edu/~emad/teaching/slides/srs_template_sep14.pdf
%http://karlwiegers.com
\documentclass[oneside,a4paper,12pt,explicit]{book}
% adapted and reformated from https://www.overleaf.com/latex/templates/msa-cs-software-requirement-specification-document/gdbzgspvjqqz
\usepackage[utf8]{inputenc}
\usepackage[sedes]{kultem}
\usepackage[english]{babel}
\usepackage{mdframed}
\usepackage{csquotes} % for compilation error
\usepackage[backend=bibtex, style=ieee]{biblatex} % You can change 'numeric' to 'ieee' or 'apa' if you want
\addbibresource{references.bib} % This points to your .bib file


%% INSERT YOUR PACKAGES HERE
\usepackage[accsupp]{axessibility}% improves PDF readability for those with disabilities.
\usepackage[hidelinks]{hyperref}
\usepackage{lipsum}
\usepackage{imakeidx}
\makeindex
\usepackage{float}
\usepackage{listings}
\usepackage{tabularx}
\setlength{\headheight}{15pt} % to avoid compiler warnings
\renewcommand{\arraystretch}{1.2} % Adjust row height for better readability
%% TITLE PAGE OPTIONS (only works with \maketitle is called in the document)
\title{Project of AAA Furnitures}
\subtitle{Software Requirement Specification}
\date{\today}
\author{Andreandhiki Riyanta Putra\\Andrian Danar Perdana\\M. Argya Vityasy}
\professor{Khabib Mustofa}
\status{DRAFT adapted from \href{https://dspmuranchi.ac.in/pdf/Blog/srs_template-ieee.pdf}{IEEE SRS Template}}
\version{1.0}

%% DOCUMENT
\begin{document}
\maketitle
\large
\tableofcontents
\normalsize
\chapter*{Revision History}

\begin{center}
    \begin{tabular}{|c|c|c|c|}
        \hline
	    Name & Date & Reason For Changes & Version\\
        \hline
	    21 & 22 & 23 & 24\\
        \hline
	    31 & 32 & 33 & 34\\
        \hline
    \end{tabular}
\end{center}

\chapter{Introduction}

\section{Purpose}
The purpose of this document is to present a detailed description of the functional
and non-functional requirements for the Furniture Selling Web Based Application.
It will also explain the app constraints, interface, and interactions with each services.
This document is intended for both the stakeholders and the developers
for a reference of developing the first version of the application.

\section{Project Scope}
AAA Furnitures is a web-based application that allows the customers of AAA Furnitures 
to purchase furnitures online. The application will allow the customers 
to browse the available furnitures, add them to the cart,
make secure payments, and track the delivery of the purchased items.
The application will also allow the admin to manage the products displayed on the website 
and the orders made by the customers.
The system will consist of multiple microservices, including the auth service, product service,
the order service, the payment service, and the delivery service.


\section{Definitions, Acronyms, and Abbreviations}
\begin{table}[H]
    \centering
    \renewcommand{\arraystretch}{1.2} % Adjust row height for better readability
    \begin{tabularx}{\textwidth}{|l|X|}
        \hline
        \textbf{Term} & \textbf{Definition} \\
        \hline
        User & A registered customer who can browse, add to cart, and place orders. \\
        \hline
        Admin & A user with special privileges to manage products, orders, and the contact for users. \\
        \hline
        Cart & A temporary collection of items selected by a user for purchase. \\
        \hline
        Order & A confirmed request for purchasing one or more items. \\
        \hline
        JWT & JSON Web Token, used for authentication and authorization. Used for the user login, logout\cite{rfc7519}  \\
        \hline
        API & Application Programming Interface, allows different services to communicate with each other.\cite{API}\\
        \hline
        REST & Representational State Transfer, an architectural style for designing APIs that use stateless communication over HTTP.\cite{REST} \\
        \hline
        Session & A temporary authentication state that maintains user login status. \\
        \hline
        SQL & Structured Query Language, used for managing relational databases by defining, querying, and modifying data.\cite{SQL}\\
        \hline
        Microservice & An architectural style that structures an application as a collection of small, independent, and loosely coupled services.\cite{9282637} \\
        \hline
        Message Queueing & A communication method where messages are sent and stored in a queue, ensuring asynchronous processing between microservices such as order, delivery, and payment services.\cite{Bouchenak2009}\\
        \hline
    \end{tabularx}
    \caption{Definitions, Acronyms, and Abbreviations}
    \label{tab:definitions}

\end{table}

\section{References}
\printbibliography[heading=none]

\chapter{Overall Description}

\section{Product Perspective}
This product is a new e-commerce platform designed as a microservices-based system. 
It aims to provide a comprehensive online shopping experience with user authentication,
product browsing, order management, payment processing, and delivery tracking capabilities.
The system is self-contained but designed with a modular architecture to allow for future expansion
and integration with third-party services such as payment gateways and shipping providers. 
The microservices architecture ensures that each component can be developed, deployed, and scaled independently.

The following figure \ref{fig:architecture} illustrates the high-level architecture of the system:

\begin{figure}[H]
    \centering
    \includegraphics[width=0.8\textwidth]{img/app diagram.drawio.png}
    \caption{System Architecture Diagram}
    \label{fig:architecture}
\end{figure}
\section{Product Functions}
The e-commerce platform will perform the following major functions:

\begin{itemize}
    \item User Authentication and Account Management
    \begin{itemize}
        \item Register new user accounts
        \item Login and logout functionality
        \item Password reset capabilities
    \end{itemize}
    
    \item Product Catalog Management
    \begin{itemize}
        \item Browse and display product information
        \item Add new products (administrator function)
        \item Delete products (administrator function)
        \item Manage product inventory
    \end{itemize}
    
    \item Order Processing
    \begin{itemize}
        \item Add products to shopping cart
        \item Place and confirm orders
        \item Cancel existing orders
        \item Track order status and delivery
    \end{itemize}
    
    \item Payment Processing
    \begin{itemize}
        \item Process secure payments for orders
        \item Notify catalog service for inventory updates
    \end{itemize}
    
    \item Delivery Management
    \begin{itemize}
        \item Arrange delivery for completed orders
        \item Track delivery status
    \end{itemize}
\end{itemize}
Figure \ref{fig:usecase} illustrates the use case for this application:
\begin{figure}[H]
    \centering
    \includegraphics[width=0.5\textwidth]{img/users diagram.drawio (1).png}
    \caption{Use Case Diagram}
    \label{fig:usecase}
\end{figure}

\section{User Classes and Characteristics}
The system will serve the following user classes:

\begin{itemize}
    \item Regular App Users
    \begin{itemize}
        \item Frequency: Regular, potentially daily use
        \item Functions: Browse products, manage cart, place orders, track deliveries
        \item Technical expertise: Minimal, should be able to navigate standard e-commerce interfaces
        \item Priority: High - primary user class
    \end{itemize}
    
    \item Administrators
    \begin{itemize}
        \item Frequency: Regular, primarily during business hours
        \item Functions: Manage product catalog, arrange deliveries, handle order issues
        \item Technical expertise: Moderate, trained on system administration
        \item Security level: High, with access to user data and system configuration
        \item Priority: Medium - essential for system operation but smaller user base
    \end{itemize}
    
    \item Guest Users
    \begin{itemize}
        \item Frequency: Occasional to one-time use
        \item Functions: Browse products only, limited functionality until registration
        \item Technical expertise: Minimal
        \item Priority: Low - encourage conversion to registered users
    \end{itemize}
\end{itemize}

\section{Operating Environment}
The software will operate in the following environment:

\begin{itemize}
    \item Server Environment
    \begin{itemize}
        \item Containerized microservices using Docker
        \item Kubernetes for orchestration
        \item Linux-based operating systems
        \item Scalable cloud infrastructure (AWS, Google Cloud, or Azure)
    \end{itemize}
    
    \item Client Environment
    \begin{itemize}
        \item Web browser support: Chrome, Firefox, Safari, Edge (latest versions)
        \item Responsive design for various screen sizes
    \end{itemize}
    
    \item Database Environment
    \begin{itemize}
        \item Dedicated database for each microservice as shown in the architecture diagram
        \item Support for SQL and NoSQL databases depending on service requirements
    \end{itemize}
    
    \item Network Environment
    \begin{itemize}
        \item HTTP/HTTPS protocols for client-server communication
        \item Message Queue (MQ) for asynchronous inter-service communication
    \end{itemize}
\end{itemize}

\section{Design and Implementation Constraints}
The following constraints will affect the design and implementation of the system:

\begin{itemize}
    \item Architectural Constraints
    \begin{itemize}
        \item Microservices architecture as depicted in the diagram
        \item RESTful API design principles for service interfaces
        \item Message Queue for asynchronous communication between services
    \end{itemize}
    
    \item Security Constraints
    \begin{itemize}
        \item Compliance with PCI DSS for payment processing
        \item Secure user authentication and authorization
        \item Data encryption for sensitive information
        \item Safe payment for user
        \item Regular security audits and penetration testing
    \end{itemize}
    
    \item Integration Constraints
    \begin{itemize}
        \item Compatible with standard payment gateways
        \item API-based integration with shipping and delivery services
    \end{itemize}
    
    \item Performance Constraints
    \begin{itemize}
        \item Response time for user interactions under 2 seconds
        \item System capable of handling at least 1000 concurrent users
        \item Scalability to accommodate peak shopping seasons
    \end{itemize}
    
    \item Development Constraints
    \begin{itemize}
        \item Coding standards and style guides for consistency
        \item Comprehensive unit and integration testing
        \item CI/CD pipeline for automated testing and deployment
    \end{itemize}
\end{itemize}

\section{Assumptions and Dependencies}
The following assumptions and dependencies apply to the project:

\begin{itemize}
    \item Assumptions
    \begin{itemize}
        \item Users have access to stable internet connections
        \item Peak usage will occur during promotional events and holidays
        \item Most users will access the system via mobile devices
        \item Inventory data will be updated in near real-time
    \end{itemize}
    
    \item Dependencies
    \begin{itemize}
        \item Reliable third-party payment processing services
        \item Shipping and delivery partner APIs
        \item Cloud infrastructure provider's uptime and service level agreements
        \item Message Queue service for inter-service communication
        \item Database management systems for each service's data storage
    \end{itemize}
    
    \item Risks
    \begin{itemize}
        \item Integration challenges with third-party services
        \item Scalability issues during peak usage periods
        \item Security vulnerabilities in payment processing
        \item Data consistency across microservices
    \end{itemize}
\end{itemize}

\chapter{External Interface Requirements}

\section{User Interfaces}
$<$Describe the logical characteristics of each interface between the software 
product and the users. This may include sample screen images, any GUI standards 
or product family style guides that are to be followed, screen layout 
constraints, standard buttons and functions (e.g., help) that will appear on 
every screen, keyboard shortcuts, error message display standards, and so on.  
Define the software components for which a user interface is needed. Details of 
the user interface design should be documented in a separate user interface 
specification.$>$

\section{Hardware Interfaces}
$<$Describe the logical and physical characteristics of each interface between 
the software product and the hardware components of the system. This may include 
the supported device types, the nature of the data and control interactions 
between the software and the hardware, and communication protocols to be 
used.$>$

\section{Software Interfaces}
$<$Describe the connections between this product and other specific software 
components (name and version), including databases, operating systems, tools, 
libraries, and integrated commercial components. Identify the data items or 
messages coming into the system and going out and describe the purpose of each.  
Describe the services needed and the nature of communications. Refer to 
documents that describe detailed application programming interface protocols.  
Identify data that will be shared across software components. If the data 
sharing mechanism must be implemented in a specific way (for example, use of a 
global data area in a multitasking operating system), specify this as an 
implementation constraint.$>$

\section{Communications Interfaces}
$<$Describe the requirements associated with any communications functions 
required by this product, including e-mail, web browser, network server 
communications protocols, electronic forms, and so on. Define any pertinent 
message formatting. Identify any communication standards that will be used, such 
as FTP or HTTP. Specify any communication security or encryption issues, data 
transfer rates, and synchronization mechanisms.$>$


\chapter{System Features}
$<$This template illustrates organizing the functional requirements for the 
product by system features, the major services provided by the product. You may 
prefer to organize this section by use case, mode of operation, user class, 
object class, functional hierarchy, or combinations of these, whatever makes the 
most logical sense for your product.$>$

\section{System Feature 1}
$<$Don’t really say “System Feature 1.” State the feature name in just a few 
words.$>$

\subsection{Description and Priority}
$<$Provide a short description of the feature and indicate whether it is of 
High, Medium, or Low priority. You could also include specific priority 
component ratings, such as benefit, penalty, cost, and risk (each rated on a 
relative scale from a low of 1 to a high of 9).$>$

\subsection{Stimulus/Response Sequences}
$<$List the sequences of user actions and system responses that stimulate the 
behavior defined for this feature. These will correspond to the dialog elements 
associated with use cases.$>$

\subsection{Functional Requirements}
$<$Itemize the detailed functional requirements associated with this feature.  
These are the software capabilities that must be present in order for the user 
to carry out the services provided by the feature, or to execute the use case.  
Include how the product should respond to anticipated error conditions or 
invalid inputs. Requirements should be concise, complete, unambiguous, 
verifiable, and necessary. Use “TBD” as a placeholder to indicate when necessary 
information is not yet available.$>$

$<$Each requirement should be uniquely identified with a sequence number or a 
meaningful tag of some kind.$>$

REQ-1:	REQ-2:

\section{System Feature 2 (and so on)}


\chapter{Other Nonfunctional Requirements}

\section{Performance Requirements}
$<$If there are performance requirements for the product under various 
circumstances, state them here and explain their rationale, to help the 
developers understand the intent and make suitable design choices. Specify the 
timing relationships for real time systems. Make such requirements as specific 
as possible. You may need to state performance requirements for individual 
functional requirements or features.$>$

\section{Safety Requirements}
$<$Specify those requirements that are concerned with possible loss, damage, or 
harm that could result from the use of the product. Define any safeguards or 
actions that must be taken, as well as actions that must be prevented. Refer to 
any external policies or regulations that state safety issues that affect the 
product’s design or use. Define any safety certifications that must be 
satisfied.$>$

\section{Security Requirements}
$<$Specify any requirements regarding security or privacy issues surrounding use 
of the product or protection of the data used or created by the product. Define 
any user identity authentication requirements. Refer to any external policies or 
regulations containing security issues that affect the product. Define any 
security or privacy certifications that must be satisfied.$>$

\section{Software Quality Attributes}
$<$Specify any additional quality characteristics for the product that will be 
important to either the customers or the developers. Some to consider are: 
adaptability, availability, correctness, flexibility, interoperability, 
maintainability, portability, reliability, reusability, robustness, testability, 
and usability. Write these to be specific, quantitative, and verifiable when 
possible. At the least, clarify the relative preferences for various attributes, 
such as ease of use over ease of learning.$>$

\section{Business Rules}
$<$List any operating principles about the product, such as which individuals or 
roles can perform which functions under specific circumstances. These are not 
functional requirements in themselves, but they may imply certain functional 
requirements to enforce the rules.$>$


\chapter{Other Requirements}
$<$Define any other requirements not covered elsewhere in the SRS. This might 
include database requirements, internationalization requirements, legal 
requirements, reuse objectives for the project, and so on. Add any new sections 
that are pertinent to the project.$>$

\chapter{Appendices}
\section{Appendix A: Glossary}
%see https://en.wikibooks.org/wiki/LaTeX/Glossary
$<$Define all the terms necessary to properly interpret the SRS, including 
acronyms and abbreviations. You may wish to build a separate glossary that spans 
multiple projects or the entire organization, and just include terms specific to 
a single project in each SRS.$>$

\section{Appendix B: Analysis Models}
$<$Optionally, include any pertinent analysis models, such as data flow 
diagrams, class diagrams, state-transition diagrams, or entity-relationship 
diagrams.$>$

\section{Appendix C: To Be Determined List}
$<$Collect a numbered list of the TBD (to be determined) references that remain 
in the SRS so they can be tracked to closure.$>$
\phantomsection
\cleardoublepage
\addcontentsline{toc}{chapter}{\indexname}
\printindex
\end{document}